\begin{tabular}{lllllll}
\toprule
 & name & color & label & bboxes & y & x \\
\midrule
0 & Lower secondary education & #FF1493 & Lower secondary education & [{'y': 49, 'x': 80, 'w': 76, 'h': 0}, {'y': 49, 'x': 156, 'w': 76, 'h': 0}, {'y': 49, 'x': 232, 'w': 76, 'h': 0}, {'y': 49, 'x': 308, 'w': 77, 'h': 0}, {'y': 49, 'x': 385, 'w': 76, 'h': 0}, {'y': 49, 'x': 461, 'w': 76, 'h': 0}, {'y': 49, 'x': 537, 'w': 76, 'h': 0}] & [11, 11, 11, 11, 11, 11, 11, 11] & [0, 1, 2, 3, 4, 5, 6, 7] \\
1 & Primary education & #00FFFF & Primary education & [{'y': 339, 'x': 80, 'w': 76, 'h': 0}, {'y': 339, 'x': 156, 'w': 76, 'h': 0}, {'y': 339, 'x': 232, 'w': 76, 'h': 0}, {'y': 339, 'x': 308, 'w': 77, 'h': 0}, {'y': 339, 'x': 385, 'w': 76, 'h': 0}, {'y': 339, 'x': 461, 'w': 76, 'h': 0}, {'y': 339, 'x': 537, 'w': 76, 'h': 0}] & [5, 5, 5, 5, 5, 5, 5, 5] & [0, 1, 2, 3, 4, 5, 6, 7] \\
\bottomrule
\end{tabular}
